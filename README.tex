\documentclass[12pt,oneside,a4paper]{article}

\usepackage[utf8]{inputenc}
\usepackage[T1]{fontenc}
\usepackage{lmodern}
\usepackage{amsmath}
\usepackage{amssymb}

\setlength{\parindent}{0mm}

\begin{document}

Chad Reynolds \hfill
CS:4640 Computer Security \\
Homework 1 \hfill
Due:  February 22, 2016

\section*{Overview}
Goal: \\
Implement two encryption algorithms out of the three possible choices:
\begin{itemize}
    \item Vigenère cipher*
    \item Permutation cipher*
    \item Simple Enigma machine with one rotor
\end{itemize}
* = my choice \\

The assignment requires the encryption to be performed on two different 
plaintext, in parts 1 and 2.  In part 1 a pre-processing step requires that 
all non-alphabetic or space characters be removed, and that the 
alphabetic characters be converted to lowercase.  In part 2 this 
pre-processing step is extended to also remove spaces. \\

See Assignment\_1.pdf for more specific details. \\

\section*{Algorithm Descriptions}
\subsection*{Vigenère}
The Vigenère cipher works in a similar fashion to the classic Caesar cipher.  
The Caesar cipher takes a character from an alphabet, then uses its index 
within that alphabet as a shift amount for each character in the plaintext. \\
Using the English alphabet as an example with the plaintext "hello", key 
of b (a shift of 1), then the ciphertext becomes "ifmmp". \\

The Vigenère cipher extends this idea by accepting a key longer than one 
character, providing a different shift amount for each character in the key.  
If the key is shorter than the plaintext, then the key is repeated a number 
of times until the entire text can be shifted.


\subsection*{Permutation}
The Permutation cipher I implemented relies solely on transposition and not 
substitution like the Vigenère.  Letters are shuffled within blocks according 
to the key.  A key $\sigma$ maps characters original positions to their new 
positions.  Again using the plaintext "hello", with the key of $\sigma$ = 
( 1 0 2 3 4 ), our plaintext becomes "ehllo". \\

We can see that all of the same letters are in place, but the character in 
index 0 has moved to index 0, and vice-versa.  The indexes can be shuffled 
around in any order in the key, keeping in mind that certain keys (such as 
this one) will not be particularly effective at obscuring the meaning of the 
plaintext.


\section*{Installation and Running instructions}
\subsection*{Installation}
The "installation" is as simple as downloading the files to a location on your 
computer.  However, the algorithms are coded in the Python programming 
language and thus depend on a Python installation being present.  
Specifically, I tested this with Python version 3.5.0 but I believe any 
Python3 installation should be capable of running the code. \\

Find up-to-date install instructions for most platforms on the Python 
website: \\
{https://www.python.org/downloads/}


\subsection*{Running the tool}
This is a command-line tool, and assumes a bash shell.  I believe the Python 
argument parsing library should run without issue in a Windows command-prompt 
but have not tested it.  There may be an issue with options. \\

To run the tool, assuming python3 is on your PATH, and the current directory 
is the project directory: \\
python3 src/main.py input\_file key\_file {vigenere | permutation} \\
(where {vigenere | permutation} is the choice of algorithm) \\

This will cause the algorithm output to be printed to standard out. There are 
options/flags for:  useful help text(-h, --help), switching to 
decryption(-d, --decrypt), and switching to part 2's no-space 
pre-processing(-n, --no-spaces)

\end{document}
